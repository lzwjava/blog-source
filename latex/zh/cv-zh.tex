\documentclass[11pt, a4paper]{latex/awesome-cv}
\usepackage{hyperref}

\usepackage{graphicx}
\usepackage{adjustbox}
\usepackage{fontspec} 	
\usepackage{xeCJK} 		
\setCJKmainfont{PingFang SC} 	
\setCJKsansfont{PingFang SC} 
\setCJKmonofont{PingFang SC} 
\setmainfont{Times New Roman}
\setromanfont{Times New Roman}
\setmonofont{Times New Roman}
\XeTeXlinebreaklocale "zh" 	

\begin{document}

\title{李智维}
\author{中国广州 · \href{mailto:lzwjava@gmail.com}{lzwjava@gmail.com} · \href{https://github.com/lzwjava}{GitHub} · \href{https://lzwjava.github.io}{作品集}}
\date{}
\maketitle

\begin{center}
    \includegraphics[width=3cm, height=3cm, keepaspectratio]{./latex/profile}
\end{center}

\textbf{简介}

高级后端/全栈工程师,拥有 11 年以上专业经验(8 年企业任职,3 年自由职业)。擅长后端系统、云原生服务、性能工程和实际机器学习项目。在构建生产系统(Java、Spring、Python)、移动应用(iOS/Android)、开发者自动化和技术写作方面拥有卓越的记录。目前通过外包合作,担任汇丰银行 (HSBC) 银行系统的后端工程师。

\textbf{教育背景}

\textbf{广东外语外贸大学} — \textit{计算机应用副学士(在读)} \\
2022 年 10 月 — 至今 \\
通过了操作系统、数据结构、高级程序设计、高等数学、数据库、计算机组成原理等 9 门课程。

\textbf{北京林业大学} — \textit{数字媒体艺术(辍学)} \\
2013 年 9 月 — 2014 年 6 月

\textbf{广州玉岩中学} — \textit{理科,中学文凭} \\
2007 年 7 月 — 2013 年 6 月

\textbf{工作经历}

\Large
\begin{itemize}
    \item \textbf{美钛技术服务(上海)有限公司 — 后端工程师(外包至汇丰银行)} \\
    \textbf{2025 年 2 月 — 至今},广州 \\
    为汇丰金融转型平台模块进行后端开发和优化。\\
    实现了财务表头的导入/验证/导出流程,并增强了提交/审批生命周期;获得了实际的会计和总账经验。\\
    独立负责端到端功能:开发 → UAT → 生产环境;编写了约 500 份供团队成员使用的技术指南。\\
    构建了自动化和垂直 Copilot 代理:约 300 个可重用脚本和 20 个专业代理,以提高生产力。\\
    \textbf{技术栈:} Java, Spring, IBM Db2, Maven, Nexus, Angular, Python, IBM WebSphere, Control-M。

    \item \textbf{自由职业者 — 全栈 / ML 工程师} \\
    \textbf{2023 年 8 月 — 2025 年 1 月} \\
    重新实现并分析了约 30 个 ML 示例项目(PyTorch, TensorFlow);完成了 DeepLearning.AI 专业化课程。\\
    使用 Flask + React 构建了一个 AI 驱动的故事机器人(Claude API);部署在 AWS 上,并使用 Prometheus + ELK 进行监控。\\
    实验了 embeddings, llama.cpp, rerankers 和 Retrieval-Augmented Generation (RAG)。\\
    维护技术博客(431 篇文章);创建了多语言翻译和 TTS 发布管道。

    \item \textbf{深圳法本信息技术股份有限公司 — 后端工程师(汇丰 PayMe)} \\
    \textbf{2022 年 11 月 — 2023 年 7 月} \\
    实现了 PayMe 自动充值:监控 Azure EventHub,处理支付后事件,应用 AOP 进行审计日志记录。\\
    参与了云迁移和 API 重构;改进了路由和安全的数据库访问。\\
    \textbf{技术栈:} Java, Spring, Kafka, Azure, AWS, Azure DevOps。

    \item \textbf{博彦科技(广州)有限公司 — 后端工程师(星展银行)} \\
    \textbf{2021 年 12 月 — 2022 年 11 月} \\
    为星展银行 Client Connect 和 DigiBank CN 构建了股票交易和共同基金微服务。\\
    实现了交易前检查,集成了 Avaloq API,自动化了 Karate 的测试生成。\\
    \textbf{技术栈:} Java, Spring Cloud, Pivotal Cloud Foundry, Kibana。

    \item \textbf{北京方根科技有限公司 — 创始人 \& 全栈工程师} \\
    \textbf{2016 年 7 月 — 2019 年 12 月} \\
    推出并运营了趣直播(知识直播平台):拥有 3 万用户,约 80 场讲座,数百万页面浏览量。全栈所有权(PHP, Vue, Laravel, Redis, RTMP)。\\
    筹集了约 50 万人民币,管理产品和运营,在咨询阶段交付了 50 个客户项目。

    \item \textbf{美味书签 (LeanCloud) — 软件工程师} \\
    \textbf{2014 年 7 月 — 2015 年 11 月} \\
    为 LeanCloud Objective-C / Java SDKs 和演示应用(LeanChat iOS/Android)做出了贡献。获得了早期云产品经验。
\end{itemize}

\textbf{证书}

\begin{itemize}
    \item 北京林业大学辍学证明
    \item 中国高等教育官方报告(本科学历,辍学状态)
    \item 雅思学术类,成绩 6 分
    \item 机器学习专业化课程结业证书,由 DeepLearning.AI 和斯坦福大学提供
    \item 深度学习专业化课程结业证书,由 DeepLearning.AI 提供
    \item AWS 开发,由 AWS 培训提供
\end{itemize}

\textbf{获奖情况}

\begin{itemize}
    \item 在 2011 年广东省 NOIP(全国青少年信息学奥林匹克联赛)初赛中,获得广州赛区一等奖;进入复赛但未获奖;可能排名前 300 名。
    \item 2014 年北京蓝桥杯大赛一等奖。
\end{itemize}

\end{document}
