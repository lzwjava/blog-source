\documentclass[a4paper,12pt]{article}
\usepackage{hyperref}
\usepackage{geometry}
\usepackage{graphicx} 
\usepackage{adjustbox}


% Configure page margins
\geometry{left=1.4cm, top=.8cm, right=1.4cm, bottom=1.8cm, footskip=.5cm}

% Enable Chinese support
\usepackage{fontspec} 	% 允許設定字體
\usepackage{xeCJK} 		% 分開設置中英文字型
\setCJKmainfont{Songti SC} 	% 設定中文字型
\setCJKsansfont{Songti SC} % Sans-serif font
\setCJKmonofont{Songti SC} % Monospace font
\setmainfont{Times New Roman}
\setromanfont{Times New Roman}
\setmonofont{Times New Roman}
\XeTeXlinebreaklocale "zh" 	% 針對中文自動換行

\linespread{1.4}\selectfont 	% 行距

\begin{document}

\title{李智维的简介}
\date{2025年1月6日}
\maketitle

\begin{center}
    \includegraphics[width=3cm, height=3cm, keepaspectratio]{/Users/lzwjava/projects/lzwjava.github.io/latex/profile} % Adjust size here
\end{center}

\Large
\begin{itemize}
    \item \textbf{基本信息}:1995年出生男性,中国公民,居住在广州。
    \item \textbf{竞赛经历}:在 2011 年广东省 NOIP 中获得前 300 名,累计解决约 1000 道算法题,链接见 \href{https://uhunt.onlinejudge.org/id/113519}{UVa Online Judge}。
    \item \textbf{教育背景}:曾在 \href{https://www.bjfu.edu.cn}{北京林业大学} 学习一年,后通过自学攻读 \href{https://www.gdufs.edu.cn}{广东外语外贸大学} 的专科学位,已完成七门课程,并为剩余的十门课程做好充分准备。
    \item \textbf{职业经历}:
        \begin{itemize}
            \item \href{https://www.teksystems.com}{TEKsystems}(外派至汇丰银行):目前正在入职后端工程师岗位,专注于汇丰企业技术部门的后端系统开发和优化。此职位涉及使用 Spring、Spring Boot、Java 核心、算法、Redis、多线程、网络、Linux 和 Google Cloud 技术。
            \item \href{https://www.farben.com.cn}{法本信息}(外派至汇丰银行):参与 \href{https://payme.hsbc.com.hk}{汇丰 PayMe} 的开发,这是一款移动支付服务,支持用户支付商家费用、点对点转账,并通过移动应用程序绑定信用卡或本地银行账户,实现便捷的日常支付场景,提升用户体验。
            \item \href{https://www.beyondsoft.com}{博彦科技}(外派至星展银行):参与 DBS Client Connect 的开发工作,该平台通过 AI 和数据驱动提升客户关系管理效率;同时负责 \href{https://www.dbs.com/digibank/in/default.page}{DBS Digibank CN} 的全新设计,为用户提供更快速、更流畅的移动银行体验,优化核心业务功能。
            \item \href{https://lzwjava.github.io/profit-en}{北京平方根}:创立知识直播平台 \textit{趣直播},一年内成功吸引 3 万用户。随后转型为软件咨询服务,先后为猿辅导、江苏电视台、粉笔教育和北京第二外国语学院等客户管理超过 50 个项目。担任项目经理和软件工程师双重角色,创造了 300 万人民币的收入和 60 万人民币的利润。
            \item \href{https://www.leancloud.cn}{LeanCloud}:负责 iOS 和 Android 开发,专注于云服务功能的 SDK 设计与实现,包括对象存储、文件存储、网站托管、容器化、即时通讯、推送通知以及短信服务,推动产品的多端无缝协作与高效运行。
        \end{itemize}
    \item \textbf{经历总结}:
        \begin{itemize}
            \item 10 年职业经验,包括 8 年企业工作经历和 2 年自由职业经历。
            \item 专业领域:
                \begin{itemize}
                    \item 移动端开发:2 年(Android 和 iOS)。
                    \item 全栈开发:7 年(6 年后端,1 年前端)。
                    \item 机器学习与大数据:1 年。
                \end{itemize}
            \item 熟悉国内外云平台,包括阿里云、\href{https://cloud.google.com}{Google Cloud}、\href{https://azure.microsoft.com}{Azure}、\href{https://aws.amazon.com}{AWS} 和腾讯云等。
        \end{itemize}
    \item \textbf{语言能力}:母语为中文,英语熟练(雅思 6 分)。
    \item \textbf{学习态度}:热衷于从基础算法到涉及数百个微服务的大型应用的学习,同时熟悉 GPT 和 Transformer 等前沿 AI 技术。注重在理论知识和实践经验之间找到平衡。
    \item \textbf{阅读经历}:阅读超过 320 本书籍,涵盖教材、个人发展、商业、技术和历史等主题。
    \item \textbf{开源贡献}:在 \href{https://github.com/lzwjava}{GitHub} 上开发了 10 个开源项目,每个项目均有超过 500 次提交。
    \item \textbf{研究经历}:自学研究近视逆转与自然视力恢复,受 Todd Becker 和 Yin Wang 的研究启发,结合三年实验结果发表了研究成果。
    \item \textbf{作品集}:参与或主导了 \href{https://lzwjava.github.io/pages/portfolio-en}{20 个项目} 的开发,涵盖初创企业项目、个人兴趣项目以及大型企业项目。
    \item \textbf{写作}:撰写了约 \href{https://lzwjava.github.io}{230 篇博客文章},内容涵盖编程、软件开发和创业等主题,每月吸引约 15,000 次页面浏览量。
    \item \textbf{生活黑客}: 创新实用解决方案,尝试使用实惠的小工具,并在女儿 2 岁时为她引入英语动画。热衷于 Arduino、Raspberry Pi Pico 和面包板等动手项目的技术爱好者。
    \item \textbf{报道}:曾被 \href{https://www.pencilnews.cn/p/13402.html}{铅笔道报道} 和 \href{https://lieyunpro.com/archives/290646}{猎云网报道}。
\end{itemize}

\end{document}
