%-------------------------------------------------------------------------------
%	SECTION TITLE
%-------------------------------------------------------------------------------
\cvsection{工作经历}

%-------------------------------------------------------------------------------
%	CONTENT
%-------------------------------------------------------------------------------
\begin{cventries}

%---------------------------------------------------------
  \cventry
    {后端工程师(入职中)} % Job title
    {美钛科技服务(上海)有限公司} % Organization
    {中国上海} % Location
    {2025年1月 - 至今} % Date(s)
    {
      \begin{cvitems} % Description(s) of tasks/responsibilities
        \item {美钛科技服务是Teksystem的子公司,而Teksystem隶属于Allegis Group, Inc.,一家跨国人才管理公司。在成功通过面试后,我收到了口头offer,目前正在入职,担任后端工程师职位,外包至汇丰银行,一家领先的全球金融机构。}
        \item {该职位将专注于在汇丰银行企业技术部门开发和优化后端系统,涉及Spring、Spring Boot、Java核心、算法、Redis、多线程、网络、Linux和Google Cloud技术。}
      \end{cvitems}
    }

%---------------------------------------------------------
  \cventry
    {后端工程师} % Job title
    {天津思芮信息技术有限公司} % Organization
    {中国广州} % Location
    {2024年8月 - 2024年10月} % Date(s)
    {
      \begin{cvitems} % Description(s) of tasks/responsibilities
        \item {思芮是一家专注于IT外包的中国高科技企业,是东软控股的子公司。Wipro是一家领先的技术服务公司,汇丰银行是一家全球金融巨头。}
        \item {成功通过面试并与思芮直接签约,通过与Wipro的合作,计划被派往汇丰银行。然而,由于汇丰银行最新的供应商政策,该职位被取消。参与了思芮和Wipro提供的培训课程。}
      \end{cvitems}
    }

%---------------------------------------------------------
  \cventry
    {自由职业者} % Job title
    {自雇} % Organization
    {中国广州} % Location
    {2023年8月 - 2024年7月} % Date(s)
    {
      \begin{cvitems} % Description(s) of tasks/responsibilities
        \item {分析并重新实现了约30个机器学习示例项目的核心部分,这些项目来自PyTorch、TensorFlow教程、Coursera在线课程或其他开源项目。获得了Coursera关于机器学习专业化和深度学习专业化的两门课程结业证书。}
        \item {为获得副学士学位相关的考试做准备,重点科目包括高等数学、计算机组成原理、线性代数等。通过听大量日语歌曲和观看大量日语TikTok视频来学习日语。}
        \item {作为全栈开发人员参与了一个基于AI的故事机器人项目,使用Claude的API生成个性化故事。该机器人支持提示设置,并包含一个管理页面用于配置。使用Python、Flask、React和Nginx开发,并部署在AWS上。使用Prometheus进行监控,ELK堆栈进行日志管理,ChatGPT-4提供编程协助。}
      \end{cvitems}
    }

%---------------------------------------------------------
  \cventry
    {后端工程师} % Job title
    {深圳法本信息技术有限公司} % Organization
    {中国广州} % Location
    {2022年11月 - 2023年7月} % Date(s)
    {
      \begin{cvitems} % Description(s) of tasks/responsibilities
        \item {法本是中国领先的软件技术服务提供商。汇丰银行是全球最大的银行和金融服务机构之一。PayMe是汇丰银行面向香港居民的移动支付服务。}
        \item {通过法本签约,被派往汇丰银行工作,并参与了PayMe项目。参与了自动充值功能的后端开发,该功能在用户余额低于一定金额时自动从其信用卡或借记卡中充值。监控来自Azure EventHub的支付后事件,并在用户设置自动充值配置时检查充值情况。使用专用的面向对象编程优雅地处理案例,并应用面向切面编程来审计自动充值表的更改日志。}
        \item {积极参与了公司AWS培训后的云迁移工作。重构API以利用基于请求头的路由,确保安全的访问和数据库配置,并参与了将微服务部署到新的云基础设施中。}
        \item {使用了包括Java、Spring和Kafka在内的强大技术栈,以及Azure、Azure DevOps和AWS用于云服务和持续集成。}
      \end{cvitems}
    }

%---------------------------------------------------------
  \cventry
    {后端工程师} % Job title
    {博彦科技咨询有限公司} % Organization
    {中国广州} % Location
    {2021年12月 - 2022年11月} % Date(s)
    {
      \begin{cvitems} % Description(s) of tasks/responsibilities
        \item {博彦科技咨询是一家领先的商业IT和咨询公司。星展银行是东南亚资产规模最大的银行,也是亚洲最大的银行之一。}
        \item {与博彦科技签约,被派往星展银行工作,并参与了DBS Client Connect和DBS DigiBank CN项目。}
        \item {在DBS Client Connect项目中,参与了股票交易微服务的开发。职责包括创建股票显示、客户显示、交易前检查和实际股票交易订单的功能。集成Avaloq API以增强底层基础设施,并通过实施编辑距离算法优化股票代码搜索的用户体验。}
        \item {在DBS DigiBank CN项目中,积极参与了多个微服务的开发,涉及共同基金管理、结构化投资产品、投资组合和交易列表。通过分析Pivotal Cloud Foundry的日志生成微服务的QPS报告,协助性能测试。开发了一个工具来自动生成测试工具Karate的测试用例,简化了测试流程并提高了测试覆盖率。}
        \item {利用了包括Java、Spring Cloud、Jira、Confluence、Jenkins、Pivotal Cloud Foundry和Kibana在内的云技术和现代框架,采用BDD和TDD方法进行自动化最佳实践。}
      \end{cvitems}
    }

%---------------------------------------------------------
\cventry
{自由职业者} % Job title
{自雇} % Organization
{中国广州} % Location
{2020年1月 - 2021年11月} % Date(s)
{
  \begin{cvitems} % Description(s) of tasks/responsibilities
    \item {撰写并发布了技术博客以在线分享知识,通过Netflix和文学作品提高英语水平,并通过解决约500个算法问题和参加Codeforces竞赛提升问题解决能力。}
    \item {通过探索入门教程和运行示例,获得了大数据和云原生技术的一些实践经验,涉及Spark、Hadoop、Kubernetes和Docker。}
    \item {完成了多个自由职业软件项目,包括LED Sign网站开发、ShowMeBug的企业微信集成、贸易数据收集的网络爬虫和名为mathjax2mobi的电子书工具。}
    \item {LED Sign网站开发(lvchensign.com):使用Bootstrap、HTML和JavaScript为一家LED标志制造公司开发了网站。实现了展示产品的功能。}
    \item {ShowMeBug的企业微信集成:参与了ShowMeBug与企业微信的集成,使技术面试工具能够在企业微信生态系统中无缝访问。使用Ruby、Ruby On Rails、PostgreSQL和微信SDK为面试官和候选人创造了流畅的用户体验。}
    \item {贸易数据收集的网络爬虫:使用Python和Selenium为一家无纺布公司开发了网络爬虫,用于收集贸易数据。自动化数据提取和页面导航,处理并将数据存储在SQLite数据库中,并生成了业务分析报告。}
    \item {mathjax2mobi:一个将带有MathJax方程的HTML内容转换为电子书友好格式的工具。通过将基于LaTeX的MathJax方程转换为SVG图像,确保与MOBI等电子书格式的兼容性。使用的技术包括Python、BeautifulSoup和Selenium。}
  \end{cvitems}
}

%---------------------------------------------------------
\cventry
{创始人兼全栈工程师} % Job title
{北京平方根科技有限公司} % Organization
{中国北京} % Location
{2016年7月 - 2019年12月} % Date(s)
{
  \begin{cvitems} % Description(s) of tasks/responsibilities
    \item {北京平方根科技有限公司在3.5年内运营了两项业务。从2016年7月到2017年9月,推出了Fun Live,一个知识直播平台。从2018年1月到2019年12月,转型为软件咨询业务。}
    \item {Fun Live允许用户参与各种知识讲座,如编程或设计。用户可以支付费用参加直播或奖励讲师。讲师使用OBS工具将直播推送到服务器。用户可以实时参与讲座或稍后观看回放。该平台与微信无缝集成以发送通知。举办了约80场讲座,获得了3万用户和数百万页面浏览量。负责大部分软件开发和营销,使用了PHP、Vue、HTML、CodeIgniter、MySQL、Redis、LeanCloud、阿里云和微信SDK。}
    \item {在软件咨询业务期间,为客户完成了50个小型软件项目,包括网站、游戏和应用程序。收入约为300万元人民币,利润约为70万元人民币。负责项目谈判、团队管理和部分软件开发。以下是一些值得注意的项目。}
    \item {面包Live:领导了面包Live的后端全面重构工作,这是一个一站式内容变现和社交经济平台。优化了整个技术栈的性能、稳定性和用户体验。之前使用ThinkPHP、Node.js和Go,并使用Laravel重写了所有服务器端开发。该平台包括课程、用户、内容、用户出勤、支付和分销销售模块。与中国顶级音频平台喜马拉雅合作,并在平台之间同步内容。}
    \item {江苏卫视《最强大脑》节目微信小程序:负责《最强大脑》节目微信小程序的所有后端开发和一半的前端开发。通过互动谜题游戏吸引观众,使他们能够竞争并排名成为“最强大脑”。使用微信小程序框架和Wepy(Vue.js)创建游戏组件和排名页面。集成RESTful API以获取游戏数据和用户信息。进行了广泛的性能调优,确保系统能够处理高并发,利用了Redis等缓存技术。}
    \item {冲顶大会:领导了冲顶大会的全栈工程开发,这是一款类似于HQ Trivia的中国移动问答应用。设计并实现了处理实时问答活动、用户管理和实时问答会话的服务和API。使用Java和Spring作为后端,Redis和Kafka用于缓存和消息队列,Zookeeper用于服务协调,Socket.IO用于实时交互。开发了管理面板以帮助操作员控制游戏。该应用支持直播、实时交互和高流量条件下的强大性能。参与了使用SEI(补充增强信息)同步直播时间戳与问答游戏交互的技术讨论。}
  \end{cvitems}
}

%---------------------------------------------------------
\cventry
{联合创始人兼全栈工程师} % Job title
{北京大米互娱有限公司} % Organization
{中国北京} % Location
{2015年11月 - 2016年7月} % Date(s)
{
  \begin{cvitems} % Description(s) of tasks/responsibilities
    \item {北京大米互娱有限公司是由包括我在内的6名互联网爱好者创立的公司。推出并运营了CodeReview平台,一个专业的代码审查、交流和分享平台。获得了约3000名用户。}
    \item {该平台包括用户管理、代码提交和审查流程、通知系统、支付集成以及活动和研讨会管理等功能。工程师可以提交他们的代码供专家审查以提高质量,专家则收取审查费用。该平台还向用户开放研讨会和活动。}
    \item {负责后端和一半前端的开发。使用了包括PHP、Vue、CodeIgniter、阿里云和Ping++在内的强大技术栈。}
  \end{cvitems}
}

%---------------------------------------------------------
\cventry
{软件工程师} % Job title
{美味书签(北京)信息技术有限公司} % Organization
{中国北京} % Location
{2014年7月 - 2015年11月} % Date(s)
{
  \begin{cvitems} % Description(s) of tasks/responsibilities
    \item {美味书签是中国领先的云计算提供商,称为LeanCloud。它提供了一套完整的云服务,包括对象存储、文件存储、Web托管、容器、即时消息、推送通知、短信和游戏后端。该公司服务了数十万开发者用户。}
    \item {参与了LeanCloud Objective-C SDK和Java SDK的开发。负责LeanChat iOS客户端和Android客户端的开发,这是一款旨在展示即时消息SDK的聊天应用程序。此外,还参与了各种前端项目。}
    \item {使用了包括iOS SDK、Android SDK、Cocoapods、Xcode、Android Studio和Angular框架在内的现代工具。}
  \end{cvitems}
}

%---------------------------------------------------------
\end{cventries}