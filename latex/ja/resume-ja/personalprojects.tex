%-------------------------------------------------------------------------------
%	セクションタイトル
%-------------------------------------------------------------------------------
\cvsection{個人プロジェクト}

%-------------------------------------------------------------------------------
%	内容
%-------------------------------------------------------------------------------
\begin{cventries}

%---------------------------------------------------------
  \cventry
    {様々な個人プロジェクト} % プロジェクト名
    {オープンソースへの貢献、ブログ開発、バックエンドおよびフロントエンド開発、モバイルアプリ開発、ツール開発} % 役割
    {} % 場所
    {2013 - 2021} % 日付
    {
      \begin{cvitems} % 説明
        \item {アルゴリズム解答: アルゴリズム問題の解答をGitHubでホストし、2466回のコミットを記録。主にJavaで実装。(2013 - 現在)}
        \item {個人ブログ: 英語と中国語のバイリンガルブログを維持し、プログラミング、技術、個人開発に関する洞察を共有。500回のコミット。(2013 - 現在)}
        \item {ライブサーバー: PHPを使用した知識ライブ配信プラットフォームのバックエンド開発。660回のコミット。(2016 - 2017)}
        \item {ライブモバイルウェブ: VueとJavaScriptを使用した知識ライブ配信プラットフォームのモバイルフロントエンド開発。528回のコミット。(2016 - 2017)}
        \item {ライブウェブ: Vueを使用した知識ライブ配信プラットフォームのデスクトップフロントエンド開発。140回のコミット。(2016 - 2017)}
        \item {ライブWeChatミニプログラム: JavaScriptを使用した知識ライブ配信プラットフォームのWeChatミニプログラム開発。63回のコミット。(2016 - 2017)}
        \item {コードレビューサーバー: PHPを使用したコードレビューのための専門プラットフォームのバックエンド開発。275回のコミット。(2015 - 2016)}
        \item {コードレビューウェブ: VueとJavaScriptを使用したコードレビューのための専門プラットフォームのフロントエンド開発。302回のコミット。(2015 - 2016)}
        \item {LeanChat iOS: LeanCloud SDKをデモンストレーションするiOSチャットアプリケーションをObjective-Cで開発。556回のコミット。(2014 - 2015)}
        \item {LeanChat Android: LeanCloud SDKをデモンストレーションするAndroidチャットアプリケーションをJavaで開発。412回のコミット。(2014 - 2015)}
        \item {ファインマン講義Mobi: LaTeXをSVGに変換してmobi電子書籍を作成するツールをPythonで開発。47回のコミット。(2020 - 2021)}
      \end{cvitems}
    }

%---------------------------------------------------------
\end{cventries}