%-------------------------------------------------------------------------------
%	セクションタイトル
%-------------------------------------------------------------------------------
\cvsection{知識}

\begin{cvparagraph}
  以下は、私の専門知識と知識領域の概要です。いくつかのトピックについては確固たる理解を持ち、他のトピックについては基本的な理解を持っています。特定の分野に興味がある場合は、面接でさらに詳しくお話しできることを嬉しく思います。
\end{cvparagraph}

%-------------------------------------------------------------------------------
%	内容
%-------------------------------------------------------------------------------
\begin{cventries}

%---------------------------------------------------------
  \cventry
    {業界知識} % カテゴリ
    {} % 役割
    {} % 場所
    {} % 日付
    {
      \begin{cvitems}
        \item {銀行と決済、教育とEラーニングプラットフォーム、ソーシャルメディアとコンテンツプラットフォーム、ユーザーとアカウント管理、通知システム、データと分析、モバイルアプリケーション、決済システム、取引管理、ソーシャルメディア統合、ビデオ管理、報酬システム、ライブビュートラッキング、アプリケーション管理、投資信託と株式取引、ブラウザベースのプログラミング環境。}
      \end{cvitems}
    }

%---------------------------------------------------------
  \cventry
    {コンピュータサイエンスの基礎} % カテゴリ
    {} % 役割
    {} % 場所
    {} % 日付
    {
      \begin{cvitems}
        \item {高等数学、コンピュータ組織、オペレーティングシステム、コンピュータネットワーク技術、データベースとアプリケーション、コンピュータ応用技術、データ構造とアルゴリズム、マイクロコンピュータとインターフェース技術。}
      \end{cvitems}
    }

%---------------------------------------------------------
  \cventry
    {技術と開発} % カテゴリ
    {} % 役割
    {} % 場所
    {} % 日付
    {
      \begin{cvitems}
        \item {多言語コミュニケーション、クロスプラットフォーム開発、フルスタックプログラミング、データベース管理、機械学習とビッグデータ、数学的熟練度、開発ツール、機械学習の実装、高度なLinuxの使用、テストと品質保証、API統合、オープンソースへの貢献、技術文書作成とブログ、クラウドコンピューティングサービス、分散システム、高性能最適化、リアルタイムアプリケーション、RTMPストリーミング。}
        \item {継続的インテグレーション/継続的デプロイメント (CI/CD)、コンテナ化とオーケストレーション、ネットワークセキュリティ、アジャイル手法、ソフトウェアアーキテクチャ、DevOpsプラクティス、クラウドネイティブアプリケーション、API開発、バージョン管理システム、サーバーレスコンピューティング、パフォーマンス監視、データエンジニアリング、セキュリティベストプラクティス、ソフトウェア開発ライフサイクル (SDLC)、技術的メンタリング、プロジェクト管理。}
      \end{cvitems}
    }

%---------------------------------------------------------
\end{cventries}