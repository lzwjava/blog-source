\documentclass[12pt]{article}
\usepackage{graphicx}
\usepackage{hyperref}  % For URLs
\usepackage{geometry}   % For page margin adjustments
\usepackage{setspace}   % For line spacing adjustments
\usepackage{parskip}    % To remove extra space between paragraphs
\usepackage{tabularx}   % To create flexible tables
\usepackage{multirow}   % For multi-row cells in tables

% Reduce line spacing and adjust margins
\linespread{1.1}
\geometry{a4paper, total={6in, 8in}}

\title{Experimental Verification of the Natural Vision Restoration Method}
\author{Li Zhiwei}
\date{June 2023}

\begin{document}

\maketitle

\begin{center}
    \includegraphics[width=0.6\textwidth]{../images/eyes/glasses.jpeg}
\end{center}

\tableofcontents

\section{Introduction}
In this paper, I present the results of an experiment that verifies the effectiveness of the natural vision restoration method inspired by the works of Todd Becker (2014) and Yin Wang (2022). The method, as proposed by Becker, involves wearing eyeglasses with a degree of myopia lower than the actual prescription, with the goal of reducing the degree of myopia over time. Yin Wang's more recent work built on Becker's foundation, offering insights into the biological mechanisms underlying this approach. Through my personal experiment conducted over the course of a year, I have observed significant improvements in both myopia and astigmatism. This paper outlines the process, results, and conclusions from my experiment.

\section{Background}
Myopia, also known as nearsightedness, is a prevalent condition that has been on the rise globally. Its causes are multifaceted, involving both genetic and environmental factors. Notably, modern lifestyles—characterized by prolonged near work (e.g., reading, screen time)—have been implicated in the increase of myopia cases. According to Todd Becker's method, first introduced in 2014, myopia can be reversed by gradually reducing the strength of corrective lenses, thereby encouraging the eye muscles to regain their natural shape.

Additionally, Yin Wang's work in 2022 further explored the methods of reversing myopia through lens adjustments and active focusing. He built upon Becker’s theory, proposing a mechanism by which reducing the strength of corrective lenses promotes the relaxation and reshaping of eye muscles, ultimately leading to improved vision. Wang’s research added a layer of understanding about the biological changes associated with this process.

An article titled ``Myopia: A Modern Yet Reversible Disease,'' presented at the 2014 Ancestral Health Symposium, highlighted that myopia is a reversible condition. The theory posits that myopia progresses in two stages: (1) near work induces lens spasm (pseudo-myopia), and (2) using minus lenses temporarily corrects the vision, leading to eye elongation and further myopia. This method of active focusing and gradual lens reduction has shown promise as a way to reverse the progression of myopia over time.

This experiment, which I began in March 2022, tests the effects of reducing myopia through a gradual reduction in lens prescriptions. Throughout this experiment, I carefully documented changes in myopia and astigmatism levels to assess the validity of this method.

\section{Methodology}

\subsection{Initial Report}
The following table presents the myopia and astigmatism measurements taken on March 5, 2022:

\begin{tabularx}{\textwidth}{|X|X|X|}
\hline
\textbf{Date} & \textbf{Left Eye (°)} & \textbf{Right Eye (°)} \\
\hline
2022.03.05 & Myopia: 350, Astigmatism: 225 & Myopia: 575, Astigmatism: 175 \\
\hline
\end{tabularx}

\subsection{Subsequent Reports}
The following tables show my myopia and astigmatism measurements on November 13, 2022, and April 20, 2023, respectively:

\begin{tabularx}{\textwidth}{|X|X|X|}
\hline
\textbf{Date} & \textbf{Left Eye (°)} & \textbf{Right Eye (°)} \\
\hline
2022.11.13 & Myopia: 325, Astigmatism: 200 & Myopia: 550, Astigmatism: 175 \\
\hline
2023.04.20 & Myopia: 300, Astigmatism: 125 & Myopia: 500, Astigmatism: 125 \\
\hline
\end{tabularx}

\section{Results}

\subsection{Myopia and Astigmatism Reduction}
Over the course of the experiment, significant improvements were observed:
\begin{itemize}
    \item My left eye's myopia reduced from 350 degrees to 300 degrees (a decrease of 50 degrees).
    \item My right eye's myopia reduced from 575 degrees to 500 degrees (a decrease of 75 degrees).
    \item Both eyes showed improvement in astigmatism, with the left eye's astigmatism reducing from 225 degrees to 125 degrees, and the right eye's astigmatism reducing from 175 degrees to 125 degrees.
\end{itemize}

These results were unexpected and highly encouraging, confirming that the natural vision restoration method had a noticeable effect.

\subsection{Eyewear Adjustment and Methodology}
To better understand the results, I carefully tracked my eyewear changes throughout the experiment. In November 2022, I switched to a new pair of glasses with 150 degrees less than my prescribed myopia, following the recommendations from Todd Becker's method.

Here is the comparison of my eyesight and eyeglasses prescription on November 13, 2022:

\begin{tabularx}{\textwidth}{|X|X|X|}
\hline
\textbf{Item} & \textbf{Left Eye (°)} & \textbf{Right Eye (°)} \\
\hline
My eyesight & Myopia: 325, Astigmatism: 200 & Myopia: 550, Astigmatism: 175 \\
Glasses I was wearing & Myopia: 225, Astigmatism: 200 & Myopia: 450, Astigmatism: 175 \\
\hline
\end{tabularx}

The reduction in myopia and astigmatism led to further improvements over the next six months. On April 20, 2023, my updated prescription was as follows:

\begin{tabularx}{\textwidth}{|X|X|X|}
\hline
\textbf{Item} & \textbf{Left Eye (°)} & \textbf{Right Eye (°)} \\
\hline
My eyesight & Myopia: 300, Astigmatism: 125 & Myopia: 500, Astigmatism: 125 \\
Glasses I was wearing & Myopia: 175, Astigmatism: 200 & Myopia: 400, Astigmatism: 175 \\
\hline
\end{tabularx}

\section{Practical Tips}
For those interested in trying the natural vision restoration method, here are some practical tips based on my experience:
\begin{itemize}
    \item \textbf{Initial Discomfort:} When you first switch to glasses with a reduced prescription (e.g., 150 degrees less), you may experience some discomfort, especially in terms of seeing distant objects clearly. However, the eyes adapt quickly, and the discomfort typically subsides within a few days.
    \item \textbf{Adjusting to Reduced Prescription:} Over time, wearing glasses with a reduced prescription for near vision becomes more comfortable and should not interfere with daily activities, such as working, studying, or using mobile devices.
    \item \textbf{Eyeglasses and Activities:} For activities such as driving, attending courses, or watching movies, you may want to consider adjusting your glasses prescription to be slightly higher. However, for everyday tasks such as using a computer or mobile phone, the reduced prescription should be sufficient.
\end{itemize}

\section{Underlying Reason and Theoretical Implications}
The underlying reason behind the success of this method remains complex and is still under research. According to Todd Becker's article, reducing the prescription of glasses encourages the eye muscles to adapt to a natural, relaxed state. While I do not fully understand the scientific mechanisms, the observed improvements in myopia and astigmatism suggest that this method is effective in some cases.

Interestingly, my experiment revealed that the recovery rate of my left and right eyes differed. It seems that my right eye, which had a higher degree of myopia, showed faster recovery, potentially due to more significant distortion in the eye muscles. This recovery pattern aligns with concepts in machine learning, where a system (in this case, the eye muscles) improves rapidly at first and then levels off over time.

\section{Conclusion}
The results of this one-year experiment suggest that the natural vision restoration method is a promising approach for reducing myopia and astigmatism. My personal experience confirms that wearing glasses with a lower prescription can lead to gradual improvements in eyesight. Although further research is needed to fully understand the mechanisms behind this method, the results so far are encouraging.

\section{Future Outlook}
I will continue monitoring my eyesight in the coming years and provide further reports. Based on the current trajectory, I expect my left eye to recover fully within three years, while my right eye may also continue to improve.

\section{References}
\begin{itemize}
    \item Becker, Todd. (2014). ``Myopia: A Modern Yet Reversible Disease.'' \url{https://gettingstronger.org/2014/08/myopia-a-modern-yet-reversible-disease}.
    \item Wang, Yin. (2022). ``Natural Vision Restoration Method.'' \url{https://www.yinwang.org/blog-cn/2022/02/22/myopia}.
\end{itemize}

\section{Citation}
Li, Zhiwei. (Jun 2023). \textit{Experimental Verification of the Natural Vision Restoration Method}. Zhiwei's Blog. \url{http://lzwjava.github.io/vision-restoration-en}

\end{document}
