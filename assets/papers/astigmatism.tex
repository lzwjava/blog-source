\documentclass[12pt]{article}
\usepackage{graphicx}
\usepackage{hyperref}  % For URLs
\usepackage[a4paper, total={6in, 8in}]{geometry}  % Adjust page margins
\usepackage{setspace}  % For line spacing
\usepackage{parskip}   % To remove extra space between paragraphs
\usepackage{tabularx}  % To create flexible tables
\usepackage{multirow}  % For multi-row cells in tables

% Adjust line spacing
\linespread{1.1}

\title{Astigmatism in Reversing Myopia}
\author{Li Zhiwei}
\date{June 2023}

\begin{document}

\maketitle

\section*{Abstract}
This paper explores the relationship between astigmatism and myopia during the process of reversing myopia. The discussion is based on the author's personal experience and a theoretical framework for how astigmatism and myopia interact during eyeball deformation and recovery. The reduction of astigmatism is hypothesized to correlate with the reversal of myopia, with astigmatism being partially reversible along with myopia.

\section{Introduction}
Myopia, or nearsightedness, is a common refractive error in the human eye. Astigmatism, another form of refractive error, is often present alongside myopia and is caused by the uneven curvature of the cornea or lens. While myopia has been well-studied, the role of astigmatism in myopia correction and reversal remains less explored. This paper aims to investigate the interaction between these two conditions during the process of myopia reversal and the potential for astigmatism reduction.

\section{Methodology}
The author has been tracking their own myopia and astigmatism progression for over a year, using a corrective eyewear method designed to reduce the curvature of the eyeball over time. The data from eye examinations between March 2022 and April 2023 are used to explore the relationship between reductions in myopia and astigmatism.

\section{Results}
The following table summarizes the eye examination results for the author’s left and right eyes over the course of the year:

\begin{table}[h!]
\centering
\begin{tabularx}{\textwidth}{|X|X|X|X|X|}
\hline
\textbf{Time} & \textbf{Myopia (Left)} & \textbf{Astigmatism (Left)} & \textbf{Myopia (Right)} & \textbf{Astigmatism (Right)} \\
\hline
2022.03 & 350 & 225 & 575 & 175 \\
2023.04 & 300 & 125 & 500 & 125 \\
\hline
\textbf{Reduction} & \textbf{50} & \textbf{100} & \textbf{75} & \textbf{50} \\
\hline
\end{tabularx}
\end{table}

From this data, we calculate the total reduction of both myopia and astigmatism in both eyes:

\begin{table}[h!]
\centering
\begin{tabularx}{\textwidth}{|X|X|X|}
\hline
\textbf{Item} & \textbf{Total Reduction (Left)} & \textbf{Total Reduction (Right)} \\
\hline
Original & 50 + 100 / 2 = 100 & 75 + 50 / 2 = 100 \\
Simplified & 100 & 100 \\
\hline
\end{tabularx}
\end{table}

We also calculate the total degrees of myopia and astigmatism:

\begin{table}[h!]
\centering
\begin{tabularx}{\textwidth}{|X|X|X|}
\hline
\textbf{Item} & \textbf{Total (Left)} & \textbf{Total (Right)} \\
\hline
Original & 300 + 125 / 2 = 362.5 & 500 + 125 / 2 = 562.5 \\
Simplified & 360 & 560 \\
\hline
\end{tabularx}
\end{table}

\section{Discussion}
It seems that the effect of astigmatism reduction is quite significant during the process of myopia reversal. Notably, the reduction in astigmatism should be considered alongside the reduction in myopia. Astigmatism, like myopia, is a measure of the eye's refractive power, and its reduction is not merely a secondary condition but an integral part of the overall process of eyeball deformation and recovery. It is suggested that the degrees of astigmatism reduction should be added to the degrees of myopia reduction because the curvature of the eye that leads to astigmatism is also involved in the development of myopia. Dividing the astigmatism degrees by two provides a reasonable approximation of its equivalent myopic reduction.

An interesting observation arises when comparing the reduction in astigmatism to myopia. For example, when the left eye’s myopia is reduced by 50 degrees and the astigmatism is reduced by 100 degrees, this is effectively equivalent to a 100-degree reduction in total refractive error. The right eye shows a similar pattern, where a 75-degree reduction in myopia, plus half the reduction in astigmatism (i.e., 50/2), adds up to an overall 100-degree reduction in refractive error. This approach underscores the interconnectedness of these two refractive conditions, with astigmatism reduction playing a pivotal role in the overall improvement in vision.

Furthermore, it is hypothesized that the amount of astigmatism reduction may be influenced by the initial severity of astigmatism. Those with more pronounced astigmatism may experience more substantial reductions due to the process of eyeball reshaping. This could reflect the timing and nature of the astigmatism’s development, which is linked to the point in time when the original deformation occurred and began to correct itself.

It is also worth noting that the specific shape of each person's eye can affect the rate and regularity of this process. The progression of myopia and astigmatism may follow a regular pattern in some individuals, leading to a more uniform form of myopia, while in others, irregularities in the eye shape may lead to the development of astigmatism. Astigmatism can be conceptualized as a columnar lens, similar to the side of a wine bottle, where the deformation creates a cylindrical shape rather than a spherical one.

\section{Conclusion}
This investigation into the relationship between astigmatism and myopia during the process of eyeball deformation and reversal supports the hypothesis that astigmatism reduction is linked to the physical process of myopia correction. While the predictions are speculative, the data suggests that both conditions may improve concurrently, with astigmatism potentially following the same trajectory as myopia reversal.

\section*{References}
\begin{itemize}
    \item Becker, Todd. (2014). ``Myopia: A Modern Yet Reversible Disease.'' \url{https://gettingstronger.org/2014/08/myopia-a-modern-yet-reversible-disease}.
    \item Wang, Yin. (2022). ``Natural Vision Restoration Method.'' \url{https://www.yinwang.org/blog-cn/2022/02/22/myopia}.
\end{itemize}

\section*{Citation}
Li, Zhiwei. (Jun 2023). \textit{Astigmatism in Reversing Myopia}. Zhiwei's Blog. \url{https://lzwjava.github.io/astigmatism-en}

Or in BibTeX:
\begin{verbatim}
@article{li2023astigmatism,
  title   = "Astigmatism in Reversing Myopia",
  author  = "Li, Zhiwei",
  journal = "lzwjava.github.io",
  year    = "2023",
  month   = "Jun",
  url     = "https://lzwjava.github.io/astigmatism-en"
}
\end{verbatim}

\end{document}
