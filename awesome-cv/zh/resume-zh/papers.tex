%-------------------------------------------------------------------------------
%	SECTION TITLE
%-------------------------------------------------------------------------------
\cvsection{论文}

\begin{cvparagraph}
  我是一名自学成才的研究者,受到Yin Wang和Todd Becker工作的启发,经过三年的实验,撰写了三篇关于逆转近视和自然视力恢复的学术风格论文。在计算机科学领域,我仍在努力取得类似的突破。
\end{cvparagraph}

%-------------------------------------------------------------------------------
%	CONTENT
%-------------------------------------------------------------------------------
\begin{cventries}

%---------------------------------------------------------
  \cventry
    {李智维} % Author
    {自然视力恢复方法的实验验证} % Paper title
    {李智维的博客} % Published in
    {2023年6月} % Date
    {
      \begin{cvitems}
        \item {讨论了验证自然视力恢复方法的实验。}
        \item {在线发布: \href{http://lzwjava.github.io/vision-restoration-en}{http://lzwjava.github.io/vision-restoration-en}}
      \end{cvitems}
    }

%---------------------------------------------------------
  \cventry
    {李智维} % Author
    {关于眼球形状恢复正常时散光的讨论} % Paper title
    {李智维的博客} % Published in
    {2023年6月} % Date
    {
      \begin{cvitems}
        \item {分析了散光与眼球形状恢复正常之间的关系。}
        \item {在线发布: \href{https://lzwjava.github.io/astigmatism-en}{https://lzwjava.github.io/astigmatism-en}}
      \end{cvitems}
    }

%---------------------------------------------------------
  \cventry
    {李智维} % Author
    {自然视力恢复:“刚好清晰”原则} % Paper title
    {李智维的博客} % Published in
    {2024年11月} % Date
    {
      \begin{cvitems}
        \item {探讨了“刚好清晰”作为自然视力恢复指导原则的概念。}
        \item {在线发布: \href{https://lzwjava.github.io/barely-clear-en}{https://lzwjava.github.io/barely-clear-en}}
      \end{cvitems}
    }

%---------------------------------------------------------
\end{cventries}