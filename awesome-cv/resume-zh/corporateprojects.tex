%-------------------------------------------------------------------------------
%	SECTION TITLE
%-------------------------------------------------------------------------------
\cvsection{企业项目}


%-------------------------------------------------------------------------------
%	CONTENT
%-------------------------------------------------------------------------------
\begin{cventries}

%---------------------------------------------------------
  \cventry
    {AI 驱动的故事机器人} % Project name
    {全栈开发} % Role
    {} % Location
    {2024年5月 - 2024年7月} % Date(s)
    {
      \begin{cvitems} % Description(s)
        \item {参与基于 Claude API 的 AI 驱动故事机器人的全栈开发,实现个性化故事。}
        \item {使用的技术包括 Python、Flask、React、AWS 和 Claude。}
      \end{cvitems}
    }

%---------------------------------------------------------
  \cventry
    {汇丰银行 PayMe} % Project name
    {后端开发} % Role
    {} % Location
    {2022年11月 - 2023年7月} % Date(s)
    {
      \begin{cvitems} % Description(s)
        \item {参与 Auto Top Up 功能和云迁移的后端开发。}
        \item {使用的技术包括 Java、Spring Cloud、Azure 和 AWS。}
      \end{cvitems}
    }

%---------------------------------------------------------
  \cventry
    {星展银行 Digibank CN} % Project name
    {后端开发} % Role
    {} % Location
    {2022年7月 - 2022年11月} % Date(s)
    {
      \begin{cvitems} % Description(s)
        \item {参与基金微服务的后端开发。}
        \item {使用的技术包括 Java、Spring Cloud 和 Pivotal Cloud Foundry。}
      \end{cvitems}
    }

%---------------------------------------------------------
  \cventry
    {星展银行 Client Connect} % Project name
    {后端开发} % Role
    {} % Location
    {2021年12月 - 2022年6月} % Date(s)
    {
      \begin{cvitems} % Description(s)
        \item {参与股票交易微服务的后端开发。}
        \item {使用的技术包括 Java、Spring Cloud 和 Pivotal Cloud Foundry。}
      \end{cvitems}
    }

%---------------------------------------------------------
  \cventry
    {ShowMeBug} % Project name
    {平台集成} % Role
    {} % Location
    {2021年7月 - 2021年9月} % Date(s)
    {
      \begin{cvitems} % Description(s)
        \item {负责平台与企业微信的集成,包括登录和便捷面试入口。}
        \item {使用的技术包括 Ruby on Rails、PostgreSQL 和 WeChat SDK。}
      \end{cvitems}
    }

%---------------------------------------------------------
  \cventry
    {Square Root Inc 项目} % Project name
    {多角色:团队管理、开发} % Role
    {} % Location
    {2018年1月 - 2019年12月} % Date(s)
    {
      \begin{cvitems} % Description(s)
        \item {参与了 50 个杂项项目,包括网站、游戏、应用程序和微信小程序。}
        \item {职责包括团队管理、合同谈判、项目管理、客户沟通和软件开发。}
      \end{cvitems}
    }

%---------------------------------------------------------
  \cventry
    {最强大脑电视节目微信小程序} % Project name
    {后端和前端开发} % Role
    {} % Location
    {2018年2月 - 2018年3月} % Date(s)
    {
      \begin{cvitems} % Description(s)
        \item {负责电视节目微信小程序的后端和前端开发。}
        \item {使用的技术包括 Java、Spring、JavaScript、HTML 和阿里云。}
      \end{cvitems}
    }

%---------------------------------------------------------
  \cventry
    {冲顶大会} % Project name
    {后端和前端开发} % Role
    {} % Location
    {2017年9月 - 2018年1月} % Date(s)
    {
      \begin{cvitems} % Description(s)
        \item {负责直播问答应用的后端开发以及管理面板前端开发。}
        \item {使用的技术包括 Java、Spring、WebSocket 和 Netty。}
      \end{cvitems}
    }

%---------------------------------------------------------
  \cventry
    {面包直播} % Project name
    {后端重构} % Role
    {} % Location
    {2017年9月 - 2018年1月} % Date(s)
    {
      \begin{cvitems} % Description(s)
        \item {负责一站式内容变现平台的后端重构。}
        \item {使用的技术包括 PHP、TypeScript、Laravel 和 Go。}
      \end{cvitems}
    }

%---------------------------------------------------------
  \cventry
    {Fun Live} % Project name
    {全栈开发} % Role
    {} % Location
    {2016年6月 - 2017年9月} % Date(s)
    {
      \begin{cvitems} % Description(s)
        \item {负责知识直播平台的大部分代码开发。}
        \item {使用的技术包括 PHP、JavaScript、HTML、阿里云和 RTMP 流媒体。}
      \end{cvitems}
    }

%---------------------------------------------------------
  \cventry
    {LeanCloud SDK 和 LeanChat} % Project name
    {SDK 和应用开发} % Role
    {} % Location
    {2014年7月 - 2015年10月} % Date(s)
    {
      \begin{cvitems} % Description(s)
        \item {负责 Objective-C 和 Java SDK 的开发,以及其演示应用 LeanChat 的全部开发。}
      \end{cvitems}
    }

%---------------------------------------------------------
\end{cventries}
