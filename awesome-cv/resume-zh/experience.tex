%-------------------------------------------------------------------------------
%	SECTION TITLE
%-------------------------------------------------------------------------------
\cvsection{工作经历}

%-------------------------------------------------------------------------------
%	CONTENT
%-------------------------------------------------------------------------------
\begin{cventries}

%---------------------------------------------------------
  \cventry
    {后端工程师} % Job title
    {天津思芮信息技术有限公司} % Organization
    {广州,中国} % Location
    {2024年8月 - 2024年10月} % Date(s)
    {
      \begin{cvitems} % Description(s) of tasks/responsibilities
        \item {思芮是一家总部位于中国的高科技企业,专注于IT外包,是东软控股的子公司。Wipro是一家领先的科技服务公司,汇丰银行是全球金融巨头。}
        \item {成功通过面试,与思芮直接签约,并通过与Wipro的合作计划被分配到汇丰银行。然而,由于汇丰银行最新的供应商政策,该职位被取消。期间参加了思芮和Wipro提供的培训课程。}
      \end{cvitems}
    }

%---------------------------------------------------------
  \cventry
    {自由职业者} % Job title
    {自由职业} % Organization
    {广州,中国} % Location
    {2023年8月 - 2024年7月} % Date(s)
    {
      \begin{cvitems} % Description(s) of tasks/responsibilities
        \item {分析并重新实现了约30个机器学习示例项目的核心部分,这些项目来源于PyTorch、TensorFlow教程、Coursera在线课程或其他开源项目。获得了Coursera的机器学习专业化和深度学习专业化课程结业证书。}
        \item {为副学士学位的考试做准备,重点学习高级数学、计算机组织学、线性代数等科目。同时,通过听大量日语歌曲和观看日本TikTok视频学习日语。}
        \item {作为全栈开发工程师参与基于Claude API的AI驱动故事机器人项目,该机器人支持prompt设置,并包含用于配置的管理页面。使用Python、Flask、React和Nginx开发,并部署在AWS上。使用Prometheus进行监控,使用ELK堆栈管理日志,并借助ChatGPT-4辅助编程。}
      \end{cvitems}
    }

%---------------------------------------------------------
  \cventry
    {后端工程师} % Job title
    {深圳法本信息技术有限公司} % Organization
    {广州,中国} % Location
    {2022年11月 - 2023年7月} % Date(s)
    {
      \begin{cvitems} % Description(s) of tasks/responsibilities
        \item {法本是中国领先的软件技术服务提供商之一,汇丰银行是全球最大的银行之一。PayMe是汇丰银行面向香港居民的移动支付服务。}
        \item {通过法本外包服务,被分配到汇丰银行工作,参与了PayMe项目的开发。主要参与了Auto Top Up功能的后端开发,该功能在用户余额低于一定金额时自动从信用卡或借记卡充值。监控Azure EventHub的支付事件,并检查用户设置的充值配置。}
        \item {积极参与云迁移工作,经过AWS培训后,重构API以利用基于请求头的路由,确保安全访问和数据库配置,并参与将微服务部署到新的云基础设施中。}
        \item {使用强大的技术栈,包括Java、Spring和Kafka,以及Azure、Azure DevOps和AWS进行云服务和持续集成。}
      \end{cvitems}
    }

%---------------------------------------------------------
  \cventry
    {后端工程师} % Job title
    {博彦科技} % Organization
    {广州,中国} % Location
    {2021年12月 - 2022年11月} % Date(s)
    {
      \begin{cvitems} % Description(s) of tasks/responsibilities
        \item {博彦科技是一家领先的业务IT和咨询公司,星展银行是东南亚最大的银行,也是亚洲最大银行之一。}
        \item {通过博彦外包服务,被分配到星展银行,参与了DBS Client Connect和DBS DigiBank CN项目。}
        \item {在DBS Client Connect项目中,参与股票交易微服务的开发,负责股票显示、客户展示、交易前检查以及下单交易功能的实现。集成Avaloq API以增强底层基础架构,并通过实现编辑距离算法优化股票代码搜索体验。}
        \item {在DBS DigiBank CN项目中,参与了多个处理共同基金管理、结构化投资产品、投资组合和交易列表的微服务开发。通过分析Pivotal Cloud Foundry的日志生成每秒查询报告,协助性能测试。开发了自动生成Karate测试用例的工具,提高了测试覆盖率。}
        \item {使用云技术和现代框架,采用BDD和TDD方法,以及包括Java、Spring Cloud、Jira、Confluence、Jenkins、Pivotal Cloud Foundry和Kibana的自动化最佳实践工具。}
      \end{cvitems}
    }

    %---------------------------------------------------------
  \cventry
  {自由职业者} % Job title
  {自由职业} % Organization
  {广州,中国} % Location
  {2020年1月 - 2021年11月} % Date(s)
  {
    \begin{cvitems} % Description(s) of tasks/responsibilities
      \item {撰写并发布技术博客,通过Netflix和文学作品提高英语水平,并通过解决约500道算法问题和参加Codeforces竞赛提升问题解决能力。}
      \item {通过探索Spark、Hadoop、Kubernetes和Docker的入门教程和示例,获得了大数据和云原生技术的基础实践经验。}
      \item {完成了多个自由职业软件项目,包括LED标牌网站开发、ShowMeBug企业微信集成、贸易数据采集网页爬虫和eBook工具mathjax2mobi。}
      \item {LED标牌网站开发(lvchensign.com):为一家LED标牌制造公司开发网站,使用Bootstrap、HTML和JavaScript实现了产品展示功能。}
      \item {ShowMeBug企业微信集成:为ShowMeBug与企业微信的集成做出了贡献,实现了在企业微信生态系统内无缝访问技术面试工具。使用技术包括Ruby、Ruby On Rails、PostgreSQL和WeChat SDK,为面试官和候选人创造了流畅的用户体验。}
      \item {贸易数据采集网页爬虫:使用Python和Selenium开发网页爬虫,为一家无纺布公司收集贸易数据。实现了自动化数据提取和页面导航,处理并将数据存储到SQLite数据库中,并生成了用于业务分析的报告。}
      \item {mathjax2mobi:一个将包含MathJax公式的HTML内容转换为电子书友好格式的工具。通过将基于LaTeX的MathJax公式转换为SVG图像,确保与MOBI等电子书格式兼容。使用的技术包括Python、BeautifulSoup和Selenium。}
    \end{cvitems}
  }

%---------------------------------------------------------
\cventry
  {创始人兼全栈工程师} % Job title
  {北京平方根信息技术有限公司} % Organization
  {北京,中国} % Location
  {2016年7月 - 2019年12月} % Date(s)
  {
    \begin{cvitems} % Description(s) of tasks/responsibilities
      \item {北京平方根信息技术有限公司在3.5年内经营了两项业务。从2016年7月到2017年9月,公司推出并运营了趣直播,一个知识直播平台。从2018年1月到2019年12月,公司转型为软件咨询业务。}
      \item {在趣直播项目中,用户可以参与各种知识讲座,如编程或设计。用户可以支付费用参加实时讲座或奖励讲师。讲师使用OBS工具将直播推流到服务器,用户可以实时参与讲座或稍后观看回放。该平台与微信集成,提供通知功能。期间共举办了约80场讲座,吸引了3万用户和数百万页面浏览量。我负责大部分软件开发和市场营销,使用的技术包括PHP、Vue、HTML、CodeIgniter、MySQL、Redis、LeanCloud、阿里云和微信SDK。}
      \item {在软件咨询业务期间,为客户完成了50个小型软件项目,包括网站、游戏和应用程序。业务收入约为300万人民币,利润约为70万人民币。我负责项目谈判、团队管理和部分软件开发。}
      \item {面包直播:负责面包直播平台的后端全面重构工作,该平台是一站式内容变现和社交经济平台。优化了整个技术栈的性能、稳定性和用户体验。原技术栈包括ThinkPHP、Node.js和Go,我将其全部重写为Laravel。平台模块包括课程、用户、内容、用户出勤、支付和分销销售。与中国领先的音频平台喜马拉雅合作,实现了内容同步。}
      \item {江苏卫视《最强大脑》微信小程序:负责《最强大脑》节目的微信小程序的全部后端开发和一半前端开发。通过互动益智游戏吸引观众,让他们竞争排名成为“最强大脑”。使用微信小程序框架和Wepy(Vue.js)创建游戏组件和排名页面。集成RESTful API以获取游戏数据和用户信息。进行了广泛的性能优化,以确保系统能处理高并发访问,使用了Redis等缓存技术。}
      \item {冲顶大会:负责冲顶大会(类似HQ Trivia)的全栈开发工作,设计并实现了直播问答事件、用户管理和实时问答的服务和API。使用Java和Spring开发后端,Redis和Kafka用于缓存和消息队列,Zookeeper用于服务协调,Socket.IO用于实时交互。开发了管理员面板,帮助运营人员管理游戏。该应用支持直播、实时互动,并在高流量条件下具有出色的性能。我还参与了技术讨论,使用SEI(补充增强信息)同步直播流的时间戳与问答游戏的交互。}
    \end{cvitems}
  }

%---------------------------------------------------------
\cventry
  {联合创始人兼全栈工程师} % Job title
  {北京大米娱乐有限公司} % Organization
  {北京,中国} % Location
  {2015年11月 - 2016年7月} % Date(s)
  {
    \begin{cvitems} % Description(s) of tasks/responsibilities
      \item {北京大米娱乐有限公司由包括我在内的6位互联网爱好者创立。推出并运营了CodeReview,一个面向代码审查、交流和分享的专业平台,共获得约3000名用户。}
      \item {平台功能包括用户管理、代码提交和审查流程、通知系统、支付集成以及活动和研讨会管理。工程师可以提交代码进行专家审查以提高其质量,专家则通过审查获得费用。平台还提供对用户开放的研讨会和活动。}
      \item {负责后端开发和一半前端开发。技术栈包括PHP、Vue、CodeIgniter、阿里云和Ping++。}
    \end{cvitems}
  }

%---------------------------------------------------------
\cventry
  {软件工程师} % Job title
  {LeanCloud (原Delicious Bookmarks)} % Organization
  {北京,中国} % Location
  {2014年7月 - 2015年11月} % Date(s)
  {
    \begin{cvitems} % Description(s) of tasks/responsibilities
      \item {LeanCloud是中国领先的云计算提供商,提供完整的云服务套件,包括对象存储、文件存储、Web托管、容器、即时通讯、推送通知、短信和游戏后端。该公司为数十万开发者用户提供服务。}
      \item {参与了LeanCloud的Objective-C SDK和Java SDK的开发。负责LeanChat iOS客户端和Android客户端的开发,这是一款旨在展示即时通讯SDK的聊天应用。此外,还参与了多个前端项目的开发。}
      \item {使用的技术包括iOS SDK、Android SDK、Cocoapods、Xcode、Android Studio和Angular框架。}
    \end{cvitems}
  }



%---------------------------------------------------------
\end{cventries}

