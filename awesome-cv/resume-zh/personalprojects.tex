%-------------------------------------------------------------------------------
% SECTION TITLE
%-------------------------------------------------------------------------------
\cvsection{个人项目} % Personal Projects

%-------------------------------------------------------------------------------
% CONTENT
%-------------------------------------------------------------------------------
\begin{cventries}

%---------------------------------------------------------
  \cventry
    {算法解决方案} % Algorithm Solutions
    {开源贡献} % Open Source Contributions
    {} % Location
    {2013 年 - 至今} % 2013 - Present
    {
      \begin{cvitems}
        \item {算法问题解决方案,托管在 GitHub 上,有 2466 次提交。}
        \item {主要使用 Java 实现。}
      \end{cvitems}
    }

%---------------------------------------------------------
  \cventry
    {个人博客} % Personal Blog
    {博客开发} % Blog Development
    {} % Location
    {2013 年 - 至今} % 2013 - Present
    {
      \begin{cvitems}
        \item {维护一个双语博客(英语和中文),有 500 次提交。}
        \item {分享关于编程、技术和个人发展的见解。}
      \end{cvitems}
    }

%---------------------------------------------------------
  \cventry
    {直播服务器} % Live Server
    {后端开发} % Backend Development
    {} % Location
    {2016 年 - 2017 年} % 2016 - 2017
    {
      \begin{cvitems}
        \item {知识直播平台的后端开发。}
        \item {使用 PHP 等技术,有 660 次提交。}
      \end{cvitems}
    }

%---------------------------------------------------------
  \cventry
    {直播移动网页} % Live Mobile Web
    {前端开发} % Frontend Development
    {} % Location
    {2016 年 - 2017 年} % 2016 - 2017
    {
      \begin{cvitems}
        \item {知识直播平台的移动前端开发。}
        \item {使用 Vue 和 JavaScript,有 528 次提交。}
      \end{cvitems}
    }

%---------------------------------------------------------
  \cventry
    {直播网页} % Live Web
    {前端开发} % Frontend Development
    {} % Location
    {2016 年 - 2017 年} % 2016 - 2017
    {
      \begin{cvitems}
        \item {知识直播平台的桌面前端开发。}
        \item {使用 Vue,有 140 次提交。}
      \end{cvitems}
    }

%---------------------------------------------------------
  \cventry
    {直播微信小程序} % Live WeChat Mini Program
    {前端开发} % Frontend Development
    {} % Location
    {2016 年 - 2017 年} % 2016 - 2017
    {
      \begin{cvitems}
        \item {知识直播平台的微信小程序开发。}
        \item {使用 JavaScript 实现,有 63 次提交。}
      \end{cvitems}
    }

%---------------------------------------------------------
  \cventry
    {代码审查服务器} % Code Review Server
    {后端开发} % Backend Development
    {} % Location
    {2015 年 - 2016 年} % 2015 - 2016
    {
      \begin{cvitems}
        \item {专业代码审查平台的后端开发。}
        \item {使用 PHP,有 275 次提交。}
      \end{cvitems}
    }

%---------------------------------------------------------
  \cventry
    {代码审查网页} % Code Review Web
    {前端开发} % Frontend Development
    {} % Location
    {2015 年 - 2016 年} % 2015 - 2016
    {
      \begin{cvitems}
        \item {专业代码审查平台的前端开发。}
        \item {使用 Vue 和 JavaScript,有 302 次提交。}
      \end{cvitems}
    }

%---------------------------------------------------------
  \cventry
    {LeanChat iOS} % LeanChat iOS
    {移动应用开发} % Mobile App Development
    {} % Location
    {2014 年 - 2015 年} % 2014 - 2015
    {
      \begin{cvitems}
        \item {开发了一个演示 LeanCloud SDK 的 iOS 聊天应用程序。}
        \item {使用 Objective-C 实现,有 556 次提交。}
      \end{cvitems}
    }

%---------------------------------------------------------
  \cventry
    {LeanChat Android} % LeanChat Android
    {移动应用开发} % Mobile App Development
    {} % Location
    {2014 年 - 2015 年} % 2014 - 2015
    {
      \begin{cvitems}
        \item {开发了一个演示 LeanCloud SDK 的 Android 聊天应用程序。}
        \item {使用 Java 实现,有 412 次提交。}
      \end{cvitems}
    }

%---------------------------------------------------------
  \cventry
    {费曼讲义 Mobi} % Feynman Lectures Mobi
    {工具开发} % Tool Development
    {} % Location
    {2020 年 - 2021 年} % 2020 - 2021
    {
      \begin{cvitems}
        \item {开发了一个将 LaTeX 转换为 SVG 以创建 mobi 电子书的工具。}
        \item {使用 Python 实现,有 47 次提交。}
      \end{cvitems}
    }

%---------------------------------------------------------
\end{cventries}