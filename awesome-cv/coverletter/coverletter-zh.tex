%!TEX TS-program = xelatex
%!TEX encoding = UTF-8 Unicode
% Awesome CV LaTeX Template for Cover Letter

%-------------------------------------------------------------------------------
% CONFIGURATIONS
%-------------------------------------------------------------------------------
\documentclass[11pt, a4paper]{../awesome-cv}

% Configure page margins
\geometry{left=1.4cm, top=.8cm, right=1.4cm, bottom=1.8cm, footskip=.5cm}

% Color for highlights
\colorlet{awesome}{awesome-red}

% Set section highlight
\setbool{acvSectionColorHighlight}{true}

% Enable Chinese support
\usepackage{fontspec} 	% 允許設定字體
\usepackage{xeCJK} 		% 分開設置中英文字型
\setCJKmainfont{PingFang SC} 	% 設定中文字型
\setCJKsansfont{PingFang SC} % Sans-serif font
\setCJKmonofont{PingFang SC} % Monospace font
\setmainfont{Times New Roman}
\setromanfont{Times New Roman}
\setmonofont{Times New Roman}
\XeTeXlinebreaklocale "zh" 	% 針對中文自動換行

\linespread{1.2}\selectfont 	% 行距

%-------------------------------------------------------------------------------
%	PERSONAL INFORMATION
%-------------------------------------------------------------------------------
\photo[circle,noedge,right]{./awesome-cv/profile}
\name{李}{智维}
\position{全栈工程师{\enskip\cdotp\enskip}后端工程师}
\address{中国,广州}

\mobile{(+86) 132-6163-0925}
\email{lzwjava@gmail.com}
\homepage{https://lzwjava.github.io}
\github{lzwjava}
\linkedin{lzwjava}

\quote{``自由与真理"}

%-------------------------------------------------------------------------------
%	LETTER INFORMATION
%-------------------------------------------------------------------------------
\recipient
  {招聘团队}
  {贵公司名称\\贵公司地址}
\letterdate{\today}
\lettertitle{全栈工程师职位申请}
\letteropening{尊敬的招聘经理,}
\letterclosing{此致,}
\letterenclosure[附件]{简历}

%-------------------------------------------------------------------------------
\begin{document}

% Header
\makecvheader[C]

% Footer
\makecvfooter{\today}{李智维~~~·~~~求职信}{}

% Print the title
\makelettertitle

%-------------------------------------------------------------------------------
%	LETTER CONTENT
%-------------------------------------------------------------------------------
\begin{cvletter}

\lettersection{关于我}
我是李智维,一名高度技能化且充满动力的全栈工程师,拥有丰富的后端开发、云计算和软件工程经验。在过去的9.5年中,我参与了多个行业的项目,包括金融科技、电商和教育等。我擅长使用Java、Spring和MySQL等技术设计并实现可扩展、可靠和高效的解决方案,同时也精通Vue和React等前端框架。

\lettersection{为什么选择贵公司?}
贵公司对创新和卓越的承诺与我的个人和职业价值观深刻契合。能够为具有重大影响力的项目贡献力量,完美符合我解决复杂问题和创造前沿解决方案的目标。我尤其被贵公司在利用现代技术推动业务增长方面的专注所打动,期待能将我的技能贡献给贵公司充满活力的团队。

\lettersection{为什么是我?}
我拥有超过7年的企业工作经验和2.5年的自由职业经历,具备独特的技术专长与创业精神。我的过往工作经历使我具备了从项目构思到部署的完整能力,能够优化后端系统并实施CI/CD管道,实现无缝集成。此外,我在AWS、Azure和阿里云等云环境中有着丰富的故障排除经验,能够有效解决贵公司面临的技术挑战。

\lettersection{主要成就}
\begin{itemize}
  \item 主导开发了一个知识直播平台,吸引了30,000名用户,举办了80多场讲座,使用PHP、MySQL和Redis技术。
  \item 参与了汇丰PayMe的自动充值功能开发,确保资金转移的无缝衔接及云迁移期间的系统可靠性。
  \item 使用Claude API、Flask和React开发了一个全面的AI故事机器人,并集成了Prometheus和ELK等现代监控与日志管理工具。
  \item 撰写技术博客并参与算法竞赛,在线解决了超过1,000道算法问题,提升了问题解决能力。
\end{itemize}

\lettersection{下一步}
我对能够为贵公司的成功做出贡献感到兴奋。期待与您进一步讨论我的技能和经验如何与贵公司的需求匹配。请查阅附上的简历,期待您的回复。

\end{cvletter}

% Print the closing
\makeletterclosing

\end{document}
