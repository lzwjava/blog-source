\documentclass[a4paper,12pt]{article}
\usepackage{hyperref}
\usepackage{geometry}
\usepackage{graphicx} 
\usepackage{adjustbox}


% Configure page margins
\geometry{left=1.4cm, top=.8cm, right=1.4cm, bottom=1.8cm, footskip=.5cm}

% Enable Chinese support
\usepackage{fontspec} 	% 允許設定字體
\usepackage{xeCJK} 		% 分開設置中英文字型
\setCJKmainfont{Songti SC} 	% 設定中文字型
\setCJKsansfont{Songti SC} % Sans-serif font
\setCJKmonofont{Songti SC} % Monospace font
\setmainfont{Times New Roman}
\setromanfont{Times New Roman}
\setmonofont{Times New Roman}
\XeTeXlinebreaklocale "zh" 	% 針對中文自動換行

\linespread{1.4}\selectfont 	% 行距

\begin{document}

\title{李智维的简介}
\date{2025年1月6日}
\maketitle

\begin{center}
    \includegraphics[width=3cm, height=3cm, keepaspectratio]{./awesome-cv/profile.png} % Adjust size here
\end{center}

\Large
\begin{itemize}
    \item \textbf{29岁男性},中国公民,居住在广州。
    \item \textbf{竞赛经历}:在 2011 年广东省 NOIP 中获得前 300 名,累计解决约 1000 道算法题,链接见 [UVa Online Judge](https://uhunt.onlinejudge.org/id/113519)。
    \item \textbf{教育背景}:曾在 [北京林业大学](https://www.bjfu.edu.cn) 学习一年,后通过自学攻读 [广东外语外贸大学](https://www.gdufs.edu.cn) 的专科学位,已完成七门课程,并为剩余的十门课程做好充分准备。
    \item \textbf{职业经历}:
        \begin{itemize}
            \item [\href{https://www.leancloud.cn}{LeanCloud}]:负责 iOS 和 Android 开发,专注于云服务功能的 SDK 设计与实现,包括对象存储、文件存储、网站托管、容器化、即时通讯、推送通知以及短信服务,推动产品的多端无缝协作与高效运行。
            \item [北京平方根](https://lzwjava.github.io/profit-en):创立知识直播平台 \textit{趣直播},一年内成功吸引 3 万用户。随后转型为软件咨询服务,先后为猿辅导、江苏电视台、粉笔教育和北京第二外国语学院等客户管理超过 50 个项目。担任项目经理和软件工程师双重角色,创造了 300 万人民币的收入和 60 万人民币的利润。
            \item [\href{https://www.beyondsoft.com}{博彦科技}](外派至星展银行):参与 [DBS Client Connect](https://www.dbs.com/newsroom/DBS_to_invest_SGD_300_million_next_year_to_further_bolster_digital_and_Intelligent_Banking_capabilities) 的开发工作,该平台通过 AI 和数据驱动提升客户关系管理效率;同时负责 [DBS Digibank CN](https://www.dbs.com/digibank/in/default.page) 的全新设计,为用户提供更快速、更流畅的移动银行体验,优化核心业务功能。
            \item [\href{https://www.farben.com.cn}{法本信息}](外派至汇丰银行):参与 [汇丰 PayMe](https://payme.hsbc.com.hk) 的开发,这是一款移动支付服务,支持用户支付商家费用、点对点转账,并通过移动应用程序绑定信用卡或本地银行账户,实现便捷的日常支付场景,提升用户体验。
        \end{itemize}
    \item \textbf{经历总结}:
        \begin{itemize}
            \item 10 年职业经验,包括 7 年企业工作经历和 3 年自由职业经历。
            \item 专业领域:
                \begin{itemize}
                    \item 移动端开发:2 年(Android 和 iOS)。
                    \item 全栈开发:7 年(6 年后端,1 年前端)。
                    \item 机器学习与大数据:1 年。
                \end{itemize}
            \item 熟悉国内外云平台,包括阿里云、[Google Cloud](https://cloud.google.com)、[Azure](https://azure.microsoft.com)、[AWS](https://aws.amazon.com) 和腾讯云等。
        \end{itemize}
    \item \textbf{语言能力}:母语为中文,英语熟练(雅思 6 分)。
    \item \textbf{学习态度}:热衷于从基础算法到涉及数百个微服务的大型应用的学习,同时熟悉 GPT 和 Transformer 等前沿 AI 技术。注重在理论知识和实践经验之间找到平衡。
    \item \textbf{阅读经历}:阅读超过 320 本书籍,涵盖教材、个人发展、商业、技术和历史等主题。
    \item \textbf{开源贡献}:在 [GitHub](https://github.com/lzwjava) 上开发了 10 个开源项目,每个项目均有超过 500 次提交。
    \item \textbf{研究经历}:自学研究近视逆转与自然视力恢复,受 Todd Becker 和 Yin Wang 的研究启发,结合三年实验结果发表了研究成果。
    \item \textbf{作品集}:参与或主导了 [20 个项目](https://lzwjava.github.io/pages/portfolio-en) 的开发,涵盖初创企业项目、个人兴趣项目以及大型企业项目。
    \item \textbf{写作}:撰写了约 [200 篇博客文章](https://lzwjava.github.io),内容涵盖编程、软件开发和创业等主题,每月吸引约 15,000 次页面浏览量。
    \item \textbf{报道}:曾被 [铅笔道报道](https://www.pencilnews.cn/p/13402.html) 和 [猎云网报道](https://lieyunpro.com/archives/290646)。
\end{itemize}

\end{document}
