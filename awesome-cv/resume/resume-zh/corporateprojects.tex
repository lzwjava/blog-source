%-------------------------------------------------------------------------------
%	SECTION TITLE
%-------------------------------------------------------------------------------
\cvsection{企业项目} % Corporate Projects

%-------------------------------------------------------------------------------
%	CONTENT
%-------------------------------------------------------------------------------
\begin{cventries}

%---------------------------------------------------------
  \cventry
    {企业项目} % Corporate Projects
    {多个项目} % Multiple Projects
    {} % Location
    {2014年7月 - 2024年7月} % 2014 - 2024
    {
      \begin{cvitems}
        \item {AI 驱动的故事机器人: 参与基于 Claude API 的 AI 驱动故事机器人的全栈开发,实现个性化故事,使用 Python、Flask、React、AWS 和 Claude。}
        \item {汇丰银行 PayMe: 参与 Auto Top Up 功能和云迁移的后端开发,使用 Java、Spring Cloud、Azure 和 AWS。}
        \item {星展银行 Digibank CN: 参与基金微服务的后端开发,使用 Java、Spring Cloud 和 Pivotal Cloud Foundry。}
        \item {星展银行 Client Connect: 参与股票交易微服务的后端开发,使用 Java、Spring Cloud 和 Pivotal Cloud Foundry。}
        \item {ShowMeBug: 负责平台与企业微信的集成,包括登录和便捷面试入口,使用 Ruby on Rails、PostgreSQL 和 WeChat SDK。}
        \item {Square Root Inc 项目: 参与了 50 个杂项项目,包括网站、游戏、应用程序和微信小程序,职责包括团队管理、合同谈判、项目管理、客户沟通和软件开发。}
        \item {最强大脑电视节目微信小程序: 负责电视节目微信小程序的后端和前端开发,使用 Java、Spring、JavaScript、HTML 和阿里云。}
        \item {冲顶大会: 负责直播问答应用的后端开发以及管理面板前端开发,使用 Java、Spring、WebSocket 和 Netty。}
        \item {面包直播: 负责一站式内容变现平台的后端重构,使用 PHP、TypeScript、Laravel 和 Go。}
        \item {趣直播: 负责知识直播平台的大部分代码开发,使用 PHP、JavaScript、HTML、阿里云和 RTMP 流媒体。}
        \item {LeanCloud SDK 和 LeanChat: 负责 Objective-C 和 Java SDK 的开发,以及其演示应用 LeanChat 的全部开发。}
      \end{cvitems}
    }

%---------------------------------------------------------
\end{cventries}
