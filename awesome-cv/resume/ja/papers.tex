%-------------------------------------------------------------------------------
%	セクションタイトル
%-------------------------------------------------------------------------------
\cvsection{論文}

\begin{cvparagraph}
  私は独学の研究者で、近視の逆転と自然な視力回復に関する3つの学術的なスタイルの論文を執筆しました。これらはYin WangとTodd Beckerの研究に触発され、3年間の実験に基づいています。コンピュータサイエンスにおいても、同様のブレークスルーを達成するために努力を続けています。
\end{cvparagraph}

%-------------------------------------------------------------------------------
%	内容
%-------------------------------------------------------------------------------
\begin{cventries}

%---------------------------------------------------------
  \cventry
    {李智維} % 著者
    {自然視力回復法の実験的検証} % 論文タイトル
    {李智維のブログ} % 出版元
    {2023年6月} % 日付
    {
      \begin{cvitems}
        \item {自然視力回復法を検証する実験について議論しました。}
        \item {オンライン公開: \href{http://lzwjava.github.io/vision-restoration-en}{http://lzwjava.github.io/vision-restoration-en}}
      \end{cvitems}
    }

%---------------------------------------------------------
  \cventry
    {李智維} % 著者
    {眼球が正常な形状に戻る際の乱視に関する考察} % 論文タイトル
    {李智維のブログ} % 出版元
    {2023年6月} % 日付
    {
      \begin{cvitems}
        \item {乱視と眼球形状の正常化との関係を分析しました。}
        \item {オンライン公開: \href{https://lzwjava.github.io/astigmatism-en}{https://lzwjava.github.io/astigmatism-en}}
      \end{cvitems}
    }

%---------------------------------------------------------
  \cventry
    {李智維} % 著者
    {自然視力回復:「かろうじてクリア」の原則} % 論文タイトル
    {李智維のブログ} % 出版元
    {2024年11月} % 日付
    {
      \begin{cvitems}
        \item {自然視力回復の指針としての「かろうじてクリア」という概念を探求しました。}
        \item {オンライン公開: \href{https://lzwjava.github.io/barely-clear-en}{https://lzwjava.github.io/barely-clear-en}}
      \end{cvitems}
    }

%---------------------------------------------------------
\end{cventries}