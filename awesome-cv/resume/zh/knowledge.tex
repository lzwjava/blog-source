%-------------------------------------------------------------------------------
%	SECTION TITLE
%-------------------------------------------------------------------------------
\cvsection{知识领域}

\begin{cvparagraph}
  以下是我专业知识和知识领域的总结。虽然我对某些主题有扎实的理解,对其他主题有基本的了解,但如果您对任何特定领域感兴趣,我将在面试中乐意分享更多信息。
\end{cvparagraph}

%-------------------------------------------------------------------------------
%	CONTENT
%-------------------------------------------------------------------------------
\begin{cventries}

%---------------------------------------------------------
  \cventry
    {行业知识} % Category
    {} % Role
    {} % Location
    {} % Date(s)
    {
      \begin{cvitems}
        \item {银行与支付、教育与电子学习平台、社交媒体与内容平台、用户与账户管理、通知系统、数据与分析、移动应用程序、支付系统、交易管理、社交媒体集成、视频管理、奖励系统、实时视图跟踪、应用程序管理、共同基金与股票交易、基于浏览器的编程环境。}
      \end{cvitems}
    }

%---------------------------------------------------------
  \cventry
    {计算机科学基础} % Category
    {} % Role
    {} % Location
    {} % Date(s)
    {
      \begin{cvitems}
        \item {高等数学、计算机组成、操作系统、计算机网络技术、数据库与应用、计算机应用技术、数据结构与算法、微机与接口技术。}
      \end{cvitems}
    }

%---------------------------------------------------------
  \cventry
    {技术与开发} % Category
    {} % Role
    {} % Location
    {} % Date(s)
    {
      \begin{cvitems}
        \item {多语言沟通、跨平台开发、全栈编程、数据库管理、机器学习与大数据、数学能力、开发工具、机器学习实现、高级Linux使用、测试与质量保证、API集成、开源贡献、技术写作与博客、云计算服务、分布式系统、高性能优化、实时应用、RTMP流媒体。}
        \item {持续集成/持续部署(CI/CD)、容器化与编排、网络安全、敏捷方法、软件架构、DevOps实践、云原生应用、API开发、版本控制系统、无服务器计算、性能监控、数据工程、安全最佳实践、软件开发生命周期(SDLC)、技术指导、项目管理。}
      \end{cvitems}
    }

%---------------------------------------------------------
\end{cventries}