%-------------------------------------------------------------------------------
%	SECTION TITLE
%-------------------------------------------------------------------------------
\cvsection{个人项目}

%-------------------------------------------------------------------------------
%	CONTENT
%-------------------------------------------------------------------------------
\begin{cventries}

%---------------------------------------------------------
  \cventry
    {各类个人项目} % 项目名称
    {开源贡献、博客开发、后端与前端开发、移动应用开发、工具开发} % 角色
    {} % 地点
    {2013 - 2021} % 日期
    {
      \begin{cvitems} % 描述
        \item {算法解决方案:算法问题的解决方案,托管在GitHub上,有2466次提交,主要使用Java实现。(2013年至今)}
        \item {个人博客:维护一个双语博客(英文和中文),有500次提交,分享编程、技术和个人发展的见解。(2013年至今)}
        \item {直播服务器:使用PHP开发的知识直播平台后端,有660次提交。(2016 - 2017年)}
        \item {直播移动网页:使用Vue和JavaScript开发的知识直播平台移动前端,有528次提交。(2016 - 2017年)}
        \item {直播网页:使用Vue开发的知识直播平台桌面前端,有140次提交。(2016 - 2017年)}
        \item {直播微信小程序:使用JavaScript开发的知识直播平台微信小程序,有63次提交。(2016 - 2017年)}
        \item {代码审查服务器:使用PHP开发的专业代码审查平台后端,有275次提交。(2015 - 2016年)}
        \item {代码审查网页:使用Vue和JavaScript开发的专业代码审查平台前端,有302次提交。(2015 - 2016年)}
        \item {LeanChat iOS:展示LeanCloud SDK的iOS聊天应用,使用Objective-C实现,有556次提交。(2014 - 2015年)}
        \item {LeanChat Android:展示LeanCloud SDK的Android聊天应用,使用Java实现,有412次提交。(2014 - 2015年)}
        \item {费曼讲座Mobi:开发将LaTeX转换为SVG以创建mobi电子书的工具,使用Python实现,有47次提交。(2020 - 2021年)}
      \end{cvitems}
    }

%---------------------------------------------------------
\end{cventries}